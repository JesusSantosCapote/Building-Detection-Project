
\documentclass[article]{llncs}
%
\usepackage[utf8]{inputenc}
\usepackage[spanish]{babel}
\usepackage{graphicx}
% Used for displaying a sample figure. If possible, figure files should
% be included in EPS format.
%
% If you use the hyperref package, please uncomment the following line
% to display URLs in blue roman font according to Springer's eBook style:
% \renewcommand\UrlFont{\color{blue}\rmfamily}


\begin{document}
%
\title{Detecci\'on de Edificaciones con Machine Learning}
%
%\titlerunning{Abbreviated paper title}
% If the paper title is too long for the running head, you can set
% an abbreviated paper title here
%
\author{Jes\'us Santos Capote, Kenny Villalobos Morales, Jorge Soler Gonz\'alez, Abraham Gonz\'alez Rivero, 
Rainel Fern\'andez Abreu, Eduardo Garc\'ia Maleta}
%
\institute{Facultad de Matemática y Computación, Universidad de La Habana, La Habana, Cuba }
%
\maketitle              % typeset the header of the contribution
%

\keywords{Detección de Objetos \and Clasificación de Im\'agenes \and Machine Learning \and Pytorch \and Keras \and Tensorflow \and Faster-RCNN}


\section{Intoducci\'on}
El uso de imágenes satelitales ha permitido avances significativos en la identificación y caracterización de diferentes 
tipos de construcciones, lo cual puede ser de gran utilidad en diversas áreas como la planificación urbana, la gestión de 
recursos naturales y la seguridad nacional, entre otras. Los autores proponen tres modelos, dos de clasificación de im\'agenes 
y uno de detección de objetos con el fin de identificar en im\'agenes satelitales edificaciones con una topolog\'ia espec\'ifica.


\section{Dataset}

\subsection{Functional Map of the World Dataset}

El dataset usado para el entrenamiento de los modelos fue el llamado Functional Map of the World Dataset. 
El conjunto de datos "Functional Map of the World" (FMOW) es un conjunto de datos público de imágenes satelitales 
que se utiliza para tareas de clasificación de objetos y detección de objetos. El conjunto de datos contiene alrededor 
de 1 millón de imágenes de alta resolución de todo el mundo, que se han etiquetado manualmente con información sobre 
las clases de objetos presentes en la imagen.

El FMOW se divide en dos partes principales: una parte de entrenamiento y una parte de prueba. La parte de entrenamiento 
consta de alrededor de 900,000 imágenes etiquetadas, mientras que la parte de prueba contiene alrededor de 100,000 
imágenes no etiquetadas. Las imágenes en el conjunto de datos muestran una variedad de paisajes y entornos, incluidas 
áreas urbanas y rurales, y se capturaron en diferentes momentos del día y en diferentes condiciones climáticas.

Las etiquetas de clase en el FMOW se basan en una taxonomía de objetos llamada "Functional Map of the World" (FMoW). La 
taxonomía FMoW se centra en las funciones que cumplen los objetos en lugar de en su apariencia física, lo que permite 
una clasificación más precisa y consistente de los objetos en diferentes contextos y entornos. Las clases de objetos 
incluyen cosas como edificios, vehículos, cuerpos de agua, cultivos y áreas verdes.

Actualmente se encuentra libre para su descarga en Amazon S3.

\section{Modelos Utilizados}

\subsection{Faster R-CNN}
Faster R-CNN es un algoritmo popular de detección de objetos que fue introducido por Shaoqing Ren, 
Kaiming He, Ross Girshick y Jian Sun en 2015. Es una extensión del modelo R-CNN original (Convolutional Neural 
Network basado en regiones), que fue introducido por Ross Girshick et al. en 2014.

La detección de objetos con Faster-RCNN se logra primero generando un conjunto de propuestas de región 
(es decir, ubicaciones de objetos candidatos) utilizando una Red de Proposición de Regiones (RPN), y luego 
clasificando estas propuestas utilizando una red de clasificación.

La RPN es una red neuronal completamente convolucional que toma una imagen como entrada y produce un conjunto de 
propuestas de objetos rectangulares, cada una con una puntuación de objetividad asociada. Estas propuestas se generan 
deslizando una pequeña red sobre el mapa de características convolucionales producido por una red de base pre-entrenada 
(típicamente una red VGG o ResNet). La RPN se entrena de extremo a extremo con la red de clasificación, utilizando una 
función de pérdida de múltiples tareas que combina una pérdida de clasificación binaria para la objetividad y una 
pérdida de regresión para las coordenadas del cuadro delimitador.

La red de clasificación toma cada propuesta generada por la RPN y realiza la clasificación de objetos y la regresión 
del cuadro delimitador. La red consta de una serie de capas completamente conectadas que toman las características de 
cada propuesta como entrada y producen una etiqueta de clase y coordenadas del cuadro delimitador.

La implementaci\'on de este modelo se realiz\'o con la utilizaci\'on de la biblioteca Pytorch de python y se realiz\'o el
entranmiento en Google Colab con la clase Stadium del dataset. Es decir el modelo se entren\'o para detectar estadios en 
im\'agenes satelitales. De igual forma se puede entrenar para detectar otro tipo de edificaci\'on.

\subsection{InceptionV3 + DNN}

El modelo de machine learning que se presenta utiliza la arquitectura InceptionV3 de redes neuronales convolucionales 
(CNN) para la extracción de características, seguida de una red neuronal densa (DNN) para la clasificación.

El modelo InceptionV3 es una red neuronal convolucional profunda que ha demostrado ser altamente eficaz en la 
clasificación de imágenes. Fue desarrollado por Google y se encuentra disponible en la librería de aprendizaje profundo 
Keras, lo que lo hace accesible para su uso en proyectos de Machine Learning.

El modelo InceptionV3 se caracteriza por su arquitectura en forma de "inception module", que se basa en la idea de 
combinar diferentes tamaños de filtros dentro de la misma capa convolucional. Esta técnica permite reducir la cantidad 
de parámetros que debe aprender el modelo, lo que a su vez reduce el riesgo de sobreajuste y mejora su capacidad de 
generalización.

Además, InceptionV3 utiliza técnicas como la regularización L2 y el dropout para prevenir el sobreajuste, y utiliza la 
función de activación ReLU para acelerar el entrenamiento de la red. El modelo también utiliza capas de agrupamiento 
máximo (max pooling) para reducir el tamaño de las características de la imagen y facilitar su procesamiento.

La red neuronal densa que se agrega a la arquitectura InceptionV3 se utiliza para la clasificación de las características 
extraídas por la red convolucional. La red consta de dos capas densas completamente conectadas con 128 y 2 neuronas respectivamente. 
La capa final de la red densa utiliza la función de activación softmax, que normaliza 
las salidas de la capa anterior para que representen probabilidades de clase para la clasificación multiclase. La función 
de activación ReLU se utiliza en la capa anterior para introducir no linealidad en la red y mejorar su capacidad de 
aprendizaje.

El modelo se entren\'o para la clasificación de im\'agenes con aeropuertos y zool\'ogicos.


\subsection{DenseNet + ResNet}

El modelo consciste en las redes neuronales convolucionales DenseNet121 y ResNet152, ambos pre-entrenados en el conjunto 
de datos de ImageNet. Las salidas de los modelos se concatenan y se alimentan a una capa completamente conectada y una 
capa de dropout para reducir el sobreajuste. Finalmente, se agrega una capa de salida con una función de activación 
softmax para realizar la clasificación en el número de clases especificado.  

DenseNet121 es una arquitectura de red neuronal convolucional profunda que se caracteriza por su alta eficiencia en el 
uso de los parámetros. En lugar de concatenar las salidas de las capas anteriores a través de capas de conexión, como en 
las redes neuronales convolucionales tradicionales, DenseNet121 conecta todas las capas de la red en una estructura 
densamente conectada. Esto significa que cada capa recibe como entrada las salidas de todas las capas anteriores, lo que 
permite que la información fluya de manera más eficiente a través de la red y evita el problema de la desaparición del 
gradiente. DenseNet121 se utiliza comúnmente en aplicaciones de clasificación de imágenes y ha demostrado un alto 
rendimiento en conjuntos de datos como ImageNet.

ResNet, por otro lado, es una arquitectura de red neuronal convolucional profunda que se caracteriza por su capacidad 
para resolver el problema de la degradación de la precisión en redes neuronales muy profundas. La degradación de la 
precisión se refiere al hecho de que a medida que se agregan más capas a una red neuronal, la precisión de la red 
comienza a disminuir en lugar de mejorar. Para resolver este problema, ResNet introduce el concepto de "conexiones de 
salto" (skip connections), que permiten que la información fluya directamente desde las capas anteriores a las capas 
posteriores, evitando que la información se pierda a medida que se profundiza en la red. ResNet se utiliza comúnmente en 
tareas de clasificación de imágenes y ha demostrado un alto rendimiento en conjuntos de datos como ImageNet.

EL modelo fue entrenado para identificar las clases: airport terminal, burial site, park, stadium, zoo. 


\begin{thebibliography}{8}
    \bibitem{citekey}
        K. He et al., “Deep residual learning for image recognition,” arXiv 1512.03385, Dec 2015.
  
    \bibitem{citekey}
        G. Huang, “Dense connected convolutional neural networks,” IEEE Computer Society Conference on Computer Vision and Pattern Recognition (CVPR), 2017.
  
  
  \end{thebibliography}

\end{document}